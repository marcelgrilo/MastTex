% ----------------------------------------------------------------
% Das conclusões tiradas
% ----------------------------------------------------------------
\chapter{Conclusão}
\label{conclusao}

Este trabalho apresenta uma abordagem evolucionária por matrizes de convolução para previsão espacial de densidades quaisquer, enfatizando em seus exemplos a densidade de carga elétrica regional. Esta abordagem usa dados históricos de pontos de instalação elétrica de diversos tipos (residencial, comercial, industrial, etc.), sendo tais dados podendo ser inseridos de qualquer fonte em uma forma normalizada pelas coordenadas geográficas, valor de densidade, tipo e  data. Quando comparado com dados históricos reais de uma cidade de tamanho médio, obtiveram resultados de alta qualidade com um erro espacial total menor que 9 \%, mesmo com acúmulo de cálculos para 3 anos consecutivos. 

A metodologia proposta utilizou quadriculas com área de 100 \(m^2\) para mapas em alta resolução, subdividos em agrupamentos planares, deta forma, obtendo um melhor impacto na previsão por adquirir características regionais mais próximas dos grupos analisados. De tal forma que o processamento e a obtenção das parâmetros de crescimento fossem obtidas através de uma otimização regional, resultando assim em diversas sub-regiões, que nos testes eram compostas de aproximadamente 130 quadriculas, podendo variar devido às bordas não serem proporcionais ao valor escolhido para a divisão das sub-regiões. Onde cada agrupamento regional tem seus próprios padrões na previsão de crescimento.

A partir deste tratamento dos mapas históricos observou-se uma função de previsão capaz de executar uma estimativa futura para um período anual \(t_{(x+1)}\), tal que esta função tenha como parâmetros um mapa de quadrículas para o ano \(t_{(x)}\) e um conjunto de informações otimizados pelo ICA, usados para formar os índices e matrizes de convolução.  

A função de previsão ainda foi adaptada para sua utilização na otimização do ICA como sendo a função de avaliação principal do problema. Assim, os parâmetros dinâmicos gerados pelas matrizes de convolução, fossem alterados e pontuados por uma \emph{regressão espacial}, onde tais parâmetros fossem modificados pela passagem desta função de avaliação por todo histórico de mapas anteriores à previsão. 

A proposta ainda demonstrou duas otimizações no ICA sobre a otimização dos valores da função de avaliação, que solucionou o problema de dimensionalidade da solução candidata pelo acrescimo de um ruído gaussiano durante a exploração do espaço de busca. A combinação dos dois métodos apresentam uma melhora ainda mais significativa na convergência para a melhor solução, como foram mostrados nos testes. 

O método proposto obteve vantagens significativas no quesito de crescimento quadrícula a quadrícula quando comparado com a previsão por curva de tendência, uma vez que as quadrículas não seguem uma função de crescimento uniforme em suas vizinhanças gerando distorções em seus índices e reduzindo a resolução do mapa futuro. Fato que não ocorre dentro da metodologia proposta com matrizes de convolução.

A maior dificuldade encontrada no trabalho foi a aquisição das informações georeferenciadas necessárias para o início dos estudos, pois mesmo com os diversos mapas disponíveis on-line, os dados de densidade populacional e a proporcional densidade de carga elétrica requer um estudo das imagens da região. Logo foi executado uma extração manual desses dados dos mapas de satélite para obter os campos de latitude, longitude, tipo de consumidor, data do mapa e densidade necessários para os testes.

O esforço computacional da metodologia com matrizes de convolução é extremamente alto, devido à busca de correlação dos mapas históricos em quadrículas para a extração dos melhores padrões em matrizes. Assim essa operação leva aproximadamente de 2 a 3 horas de processamento paralelo para todas as 225 regiões do teste, em um computador com Windows 10 e processador AMD-FX8350 de 8 núcleos com clock fixo de 3GHz e 16GB RAM, calculando a matriz de convolução de ordem 7 com 3 ponderações. Como o código do ICA foi desenvolvido de forma a ser otimizado em relação às suas operações internas, 93\% do tempo total é ocasionado pelo cálculo paralelo da função de avaliação, que realiza inúmeras operações matriciais com os mapas históricos de quadrículas. Como exemplo, o número de operações de convolução que ocorrem durante uma década é igual ao número de colônias vezes o número de ponderações escolhidas. É importante salientar que mesmo com um alto tempo de cálculo a metodologia proposta é paralela, permitindo sua utilização em núvens de clusters de computadores, reduzindo de forma abrupta esse tempo de cálculo obtido nos experimentos.

O destaque da proposta fica evidente pela qualidade dos resultados gerados, tanto pela alta resolução quanto pelo alto grau de acerto, obtidos pelas contribuições no uso de matrizes de convolução para identificação e cálculo dos padrões de crescimento das áreas, quanto a eficiência e robustez do ICA para a busca dessas matrizes em um problema tão grande e complexo em proporções.

A aplicação desses resultados são de grande valia para as empresas concessionárias de energia, pois com a qualidade de valores futuros e a grande resolução de seus mapas permitem o planejamento da expansão do sistema elétrico de um município, identificando com precisão locais de novas subestações, instalação de novos transformadores e a inserção de acumuladores com baterias de sal em locais estratégicos, aumentando a confiabilidade da rede e reduzindo custos.

Inicialmente proposta para o emprego em Smartgrids, o previsor espacial urbano durante os estudos demonstrou um emprego muito mais amplo além do foco em mercados consumidores de energia por sua resolução mais apurada nos mapas futuros. Planejamentos urbanos de vias, serviços fundamentais como escolas e clínicas, fornecimento de comunicação, água e gás podem utilizar sem nenhuma modificação a metodologia proposta, demonstrando ser uma ferramenta poderosa para a solução dos principais problemas urbanos encontrados em nossas cidades. 

\section{Colaborações do Trabalho}
\label{colaborações_do_trabalho}

% ICA modificado + matriz de convolução para identificação de características do mapa %

Este trabalho apresenta duas colaborações relevantes, sendo a primeira a resolução do problema da dimensionalidade dos limites dos atributos do ICA, que fora solucionado com duas implementações distintas, podendo elas serem utilizadas separadamente ou de forma combinada, alterando diretamente a forma como os ruídos são inseridos durante a assimilação de colônias por um império, diminuindo o tempo de conversão, em número de décadas, para o mínimo global. Além disso, o ICA foi desenvolvido de forma a ser facilmente utilizado por qualquer aplicação, isolando a sua lógica do processo evolutivo do problema a ser implementado, de modo que, qualquer problema que venha a ser modelado no ICA não precise alterar os componentes relativos ao processo evolutivo. 

A segunda colaboração deste trabalho é referente à forma como foram usadas as operações de convolução, sendo estas combinadas em uma função de previsão, que faz ponderações sobre diversas matrizes de convolução, que neste caso são treinadas pelo ICA com o intuito de obter características de crescimento de densidade distintas de uma mesma região, as quais, junto de outros atributos, são usadas para formar um mapa futuro de previsão. Com a operação de convolução, é possível buscar tais características de crescimento para cada ponto de modo que tanto o ponto em questão quanto os pontos vizinhos sejam levados em consideração no momento da definição da característica, durante a etapa de treino. A ponderação entre as diversas matrizes de convolução garante que a previsão de um dado ponto condiz mais com a realidade, dentre diversas características.

\section{Trabalhos Futuros}
\label{trabalhos_futuros}

Este trabalho aplicou a função de previsão sobre os dados de densidade de carga elétrica. Porém, tal técnica pode ser utilizada para a busca de um mapa de crescimento futuro baseado em qualquer tipo de densidade. Para trabalhos futuros tem-se a intenção de alterar a função de previsão para que esta faça o uso de outra técnica de processamento de imagens ou até mesmo, que venha a utilizar técnicas de visão computacional, como a identificação de padrões através de técnicas do tipo Haar \cite{mita2005joint}, que geralmente são utilizadas para encontrar padrões faciais, porém podem ser aplicadas para o treinamento e reconhecimento de qualquer tipo de padrão em imagens, sendo possível a detecção de padrões de crescimento sobre a identificação de padrões na forma e intensidade dos pontos da imagem, diminuindo a quantidade de parâmetros, porém aumentando a quantidade de características de crescimento de uma região.


% haar-feature para identificação de padrões para regiões de crescimento %
