% ----------------------------------------------------------------
% Das conclusões tiradas
% ----------------------------------------------------------------
\chapter{Conclusão}
\label{conclusao}

Este trabalho apresenta uma abordagem evolucionaria para previsão espacial de densidades quaisquer, enfatizando no trabalho e em seus exemplos e aplicações a densidade de carga elétrica. Esta abordagem usa dados históricos de pontos de instalação elétrica de diversos tipos (residencial, comercial, industrial, etc.), sendo tais dados podendo ser inseridos de qualquer fonte em uma forma normalizada (coordenada geográfica, valor de densidade, tipo e  data). Quando comparado com dados históricos reais de uma cidade de tamanho médio, os resultados apresentados foram de alta qualidade, apresentando um erro espacial de menos de 9 \% para uma previsão que acumulou erros durante 3 períodos consecutivos. 

São gerados então diversos mapas históricos, isto é, que representam como cada uma dentre todas as suas pequenas regiões se encontrava, quantitativamente em relação à um parâmetro principal (densidade de carga, densidade populacional, etc.) a ser analisado, parâmetro este, que por sua vez, pode ser composto por outros fatores (no caso da densidade de carga: residencial, comercial etc.). 

Para que o método proposto tivesse um impacto melhor na previsão e adquirisse mais características de regiões mais próximas do ponto a ser analisado, o mapa de quadrículas foi criado em alta resolução (quadrículas de 100 \(m^2\) ), tal que o processamento e a obtenção das características de crescimento fossem obtidas através da otimização regional, formando assim diversas sub regiões (compostas de aproximadamente 130 quadriculas, algumas maiores que outras devido às bordas não serem proporcionais ao valor escolhido para a divisão das sub regões), onde cada sub região tem seus próprios parâmetros de previsão de crescimento. Assim cada mapa histórico é composto por sub regiões, que por sua vez são compostas por quadrículas.   
A partir da observação e análise destes mapas históricos imaginou-se uma função de previsão capaz de executar uma previsão futura de um período\(t_{x+1}\), tal que esta função tenha como parâmetros, um mapa de quadrículas (sub região de quadriculas) de um dado período \(t_{x}\) e um conjunto de parâmetros (otimizados pelo ICA, usados para formar os fatores e matrizes de convolução).  

Após a concepção da função de previsão, esta fora adaptada de modo que pudesse ser utilizada como função de avaliação do ICA, tal que seus parâmetros dinâmicos (todos os parâmetros com exceção do mapa de quadrículas) fossem otimizados na forma de uma \emph{regressão espacial}, sendo que tais parâmetros ficassem otimizados pela passagem da função de previsão por todo histórico de mapas anteriores à previsão. 

Foram propostas duas otimizações no algoritmo que faz a otimização dos parâmetros da função de previsão, solucionando um problema de dimensionalidade ao se adicionar ruído durante a exploração do espaço de busca da solução. A combinação dos dois métodos apresenta uma melhora ainda mais significativa na convergência para a melhor solução. 

O método proposto é ainda comparado com o método estatístico de previsão por curva de tendência, onde o método proposto obteve vantagem em tal comparação no quesito de comparação ponto a ponto, uma vez regiões não seguem uma função de crescimento linear, e obteve a mesma taxa de acerto em relação ao crescimento da região, o que valida esta abordagem proposta.

Uma possível desvantagem deste método é a aquisição dos dados reais, que podem se fazer complexos, por isto inicialmente resume-se a base de dados em latitude, longitude, tipo, data e valor. A representação do problema com dados reais de carga se fazem importantes para melhorar a qualidade da previsão e executar outras atividades relacionadas ao planejamento e operações de sistemas de distribuição. 

Outra desvantagem aparente é o tempo de processamento da função de convolução, sendo esta uma operação muito pesada, levando aproximadamente 2 a 3 horas para o processamento de todas as 225 regiões em um computador AMD-FX8350 3gHz (limitado) e 16GB RAM usando a matriz de convolução de ordem 7 e 3 ponderações e com o processamento paralelo ligado. Como o código do ICA foi desenvolvido de forma a ser ótimo em relação às suas operações internas, a maior parte do tempo se deve ao processamento da função de avaliação, e não à rotinas do ICA. Pois na Função de avaliação ocorrem, neste caso, uma operações de convolução de imagem (em escala de cinza) para ponderação durante cada período a ser analisado na otimização, sendo que o número de operações de convolução que podem ocorrer durante uma 

Uma das grandes vantagens deste método sobre os demais é a consideração dos diversos pontos vizinhos durante a etapa de otimização dos atributos para a função de previsão, e devido à operação de ponderação dos resultados de cada pixel calculado de convolução, executada na função de avaliação faz com que os valores tenham uma precisão maior e condizem mais realmente que métodos estatísticos com a realidade. 

Os resultados deste tipo de previsão são de grande valor para as concessionárias, por exemplo, em um caso de planejamento de expansão, será possível encontrar os pontos ideais para inserir novos transformadores ou realocar transformadores usados e outros dispositivos para outras partes da cidade economizando dinheiro, podendo diminuir os investimentos em novos equipamento e otimizando o uso de equipamentos sub-utilizados.

Com esta previsão de crescimento de alta definição é possível então fazer o planejamento sustentável, em planejamento de expansão urbana em diversos aspectos, como por exemplo, de sistemas de \emph{smartgrids} em regiões urbanas com custo mínimo de implantação, ou seja, quando se sabe onde e como uma cidade irá crescer é possível baratear os custos de implantação, direcionando investimentos para regiões mais aptas, melhorando o consumo geral.

Em resumo, este trabalho resultou em uma metodologia de previsão de densidade de áreas urbanas capaz de manter uma alta definição na previsão espacial de forma bidimensional (mapas de quadrículas em 2d),  seguindo a taxa de crescimento padrão segundo as metodologias de previsão existentes para regiões urbanas.


\section{Colaborações do Trabalho}
\label{colaborações_do_trabalho}

% ICA modificado + matriz de convolução para identificação de características do mapa %

Este trabalho apresenta duas colaborações relevantes, sendo a primeira a resolução do problema da dimensionalidade dos limites dos atributos do ICA, que fora solucionado com duas implementações distintas, podendo elas serem utilizadas separadamente ou de forma combinada, alterando diretamente a forma como os ruídos são inseridos durante a assimilação de colônias por um império, diminuindo o tempo de conversão, em número de décadas, para o mínimo global. Além disso, o ICA foi desenvolvido de forma a ser facilmente utilizado por qualquer aplicação, isolando a sua lógica do processo evolutivo do problema a ser implementado, de modo que, qualquer problema que venha a ser modelado no ICA não precise alterar os componentes relativos ao processo evolutivo. 

A segunda colaboração deste trabalho é referente à forma como foram usadas as operações de convolução, sendo estas combinadas em uma função de previsão, que faz ponderações sobre diversas matrizes de convolução, que neste caso são treinadas pelo ICA com o intuito de obter características de crescimento de densidade distintas de uma mesma região, as quais, junto de outros atributos, são usadas para formar um mapa futuro de previsão. Com a operação de convolução, é possível buscar tais características de crescimento para cada ponto de modo que tanto o ponto em questão quanto os pontos vizinhos sejam levados em consideração no momento da definição da característica, durante a etapa de treino. A ponderação entre as diversas matrizes de convolução garante que a previsão de um dado ponto condiz mais com a realidade, dentre diversas características (matrizes de convolução).

\section{Trabalhos Futuros}
\label{trabalhos_futuros}

Este trabalho aplicou a função de previsão sobre os dados de densidade de carga elétrica. Porém, tal técnica pode ser utilizada para a busca de um mapa de crescimento futuro baseado em qualquer tipo de densidade. Para trabalhos futuros tem-se a intenção de alterar a função de previsão para que esta faça o uso de outra técnica de processamento de imagens ou até mesmo, que venha a utilizar técnicas de visão computacional, como a identificação de padrões através de técnicas do tipo Haar \cite{mita2005joint}, que geralmente são utilizadas para encontrar padrões faciais, porém podem ser aplicadas para o treinamento e reconhecimento de qualquer tipo de padrão em imagens, sendo possível a detecção de padrões de crescimento sobre a identificação de padrões na forma e intensidade dos pontos da imagem, diminuindo a quantidade de parâmetros, porém aumentando a quantidade de características de crescimento de uma região.


% haar-feature para identificação de padrões para regiões de crescimento %
