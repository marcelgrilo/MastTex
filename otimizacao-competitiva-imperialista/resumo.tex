\setlength{\absparsep}{18pt} % ajusta o espaçamento dos parágrafos do resumo
\begin{resumo}

A previsão de crescimento espacial apresenta um desafio em diversos aspectos, sejam eles computacionais, de planejamento futuro ou tipo de dado a ser analisado, pelo grande volume de dados a serem tratados, incertezas a serem consideradas e por sua utilização frequente em planejamentos futuros de áreas urbanas. A expansão espacial encontrada pelo estudo de previsão são empregadas no planejamento populacional, elétrico ou hídrico de uma região administrativa. Assim é apresentada uma metodologia de previsão espacial de áreas urbanas em alta resolução para determinar a densidade futura no planejamento de médio prazo. Esta metodologia faz o emprego de mapas regionais subdivididos em quadrículas, em que cada mapa serão calculadas matrizes de convolução para representar os padrões de crescimento anual dos blocos de quadrículas dessa área. Logo, ao aplicar essas matrizes de convolução numa região de quadrículas, será gerado um novo mapa deste local um ano a frente, método similar ao filtro computacional de imagens por convolução. Mas a busca pelas matrizes de convolução é complexa e penosa. Assim é proposto um algoritmo imperialista competitivo modificado para a busca dos parâmetros das matrizes de convolução, comparando seus resultados com densidades históricas do mapa em estudo. A natureza dos dados utilizados nos testes da metodologia são direcionados para as cargas elétricas de uma cidade. Assim é possível obter uma previsão da densidade crescendo tanto em intensidade quanto em espaço, em alta resolução, a qual poderá ter várias aplicações, podendo ser usada como importante fator no planejamento de redes elétricas inteligentes (Smartgrids), planejamento de expansão de cidades, esgotos, redes de telecomunicações, entre outros. 

\end{resumo}

\begin{resumo}[Abstract]
 \begin{otherlanguage*}{english}

The prediction of spatial growth presents a challenge in several aspects, be they computational, future planning or type of data to be analyzed, by the large volume of data to be treated, uncertainties to be considered and their frequent use in future planning of urban areas. The spatial expansion found by the prediction study is used in the population, electric or water planning of an administrative region. Thus, a methodology of spatial prediction of urban areas in high resolution is presented to determine the future density in the medium term planning. This methodology makes use of regional maps subdivided into squares, in which each map will be calculated convolution matrices to represent the annual growth patterns of the squared blocks of that area. Therefore, when applying these convolution matrices in a region of squares, a new map of this location will be generated one year ahead, using a method similar to the computational filter of convolution images. But the search for convolution matrices is complex and painful. Thus, a modified competitive imperialist algorithm is proposed for the search of the parameters of the convolution matrices, comparing its results with historical densities of the map under study. The nature of the data used in the methodology tests are directed to the electric charges of a city. Thus, it is possible to obtain a density prediction increasing in both intensity and space, in high resolution, which may have several applications, and can be used as an important factor in the planning of intelligent electrical networks (Smartgrids), urban expansion planning, sewers, telecommunications, among others.



 \end{otherlanguage*}
\end{resumo}