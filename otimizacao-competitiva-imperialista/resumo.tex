\setlength{\absparsep}{18pt} % ajusta o espaçamento dos parágrafos do resumo
\begin{resumo}

A previsão de crescimento espacial apresenta um desafio computacional pelo grande volume de dados a serem tratados e por sua utilização frequente em planejamentos futuros de áreas urbanas. A expansão espacial encontrada pelo estudo de previsão são empregadas no planejamento populacional, elétrico ou hídrico de uma região administrativa. Assim é apresentada uma metodologia de previsão espacial de áreas urbanas em alta resolução para determinar a densidade futura no planejamento de médio prazo. Esta metodologia faz o emprego de mapas regionais subdivididos em quadrículas, onde em cada mapa serão calculadas matrizes de convolução para representar os padrões de crescimento anual dos blocos de quadrículas dessa área. Logo, ao aplicar essas matrizes de convolução numa região de quadrículas, será gerado um novo mapa deste local um ano a frente, método similar ao filtro computacional de imagens por convolução. Mas a busca pelas matrizes de convolução é complexa e penosa. Assim foi aplicado um algoritmo imperialista competitivo modificado para a busca dos parâmetros das matrizes de convolução, comparando seus resultados com densidades históricas do mapa em estudo. A natureza dos dados utilizados nos testes da metodologia são direcionados para as cargas elétricas de uma cidade. Assim é possível obter uma previsão da densidade de consumo de energia de alta resolução que poderá ser usada como importante fator no planejamento de redes elétricas inteligentes nessas regiões. 

\end{resumo}

\begin{resumo}[Abstract]
 \begin{otherlanguage*}{english}

The prediction of spatial growth presents a computational challenge due to the large volume of data to be processed and its frequent use in future urban planning. The spatial expansion found by the prediction study are used in the population, electric or water planning of an administrative region. Thus, a methodology of spatial prediction of urban areas in high resolution is presented to determine the future density in the medium term planning. This methodology makes use of regional maps subdivided into squares, where in each map convolution matrices will be calculated to represent the annual growth patterns of the squared blocks of that area. Then, when applying these convolution matrices in a region of squares, a new map of this location will be generated one year ahead, a method similar to the convolution image computational filter. But the search for convolution matrices is complex and painful. Thus, a modified competitive imperialist algorithm was applied to search the parameters of the convolution matrices, comparing their results with historical densities of the map under study. The nature of the data used in the methodology tests are directed to the electric charges of a city. Thus, it is possible to obtain a prediction of the density of energy consumption of high resolution that can be used as important factor in the planning of Smartgrids in these regions.

 \end{otherlanguage*}
\end{resumo}