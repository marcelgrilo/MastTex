\setlength{\absparsep}{18pt} % ajusta o espaçamento dos parágrafos do resumo
\begin{resumo}

A previsão espacial de crescimento de densidades em regiões urbanas apresenta um grande desafio por ser muito utilizada em planejamentos de expansão de áreas urbanas, sendo estas densidades relacionadas ao tipo de planejamento de expansão que se deseja fazer, carga elétrica, populacional, hídrico, ou qualquer outro. Apresenta-se então uma metodologia para previsão de áreas urbanas em alta resolução, a qual melhora a visualização, análise e inferência das informações de densidade para o planejamento em um futuro próximo. Tal metodologia converte os dados de entrada em mapas de quadrículas de alta resolução, que por sua vez são divididos em regiões maiores, onde cada uma destas regiões terá seu crescimento esperado definido por matrizes de convolução e fatores de ponderação, que buscam por características históricas de cada região. A obtenção das melhores características de crescimento das regiões vem através do processamento do algoritmo imperialista competitivo buscar o melhor conjunto de parâmetros, onde tais parâmetros são compostos pelos valores das matrizes de convolução e dos fatores de ponderação e serão utilizados para prever o crescimento da região. A natureza dos dados utilizados nos testes direciona a previsão para a densidade de carga elétrica, assim é possível obter uma previsão espacial de crescimento em alta resolução e maior precisão que, neste caso, pode ser usada como um importante fator no planejamento de expansão de carga, podendo até mesmo ser usado no planejamento e aplicações dos conceitos de \emph{smartgrids} (redes elétricas inteligentes) na área urbana definida.

\end{resumo}

\begin{resumo}[Abstract]
 \begin{otherlanguage*}{english}

The spatial forecasting of density growth in urban regions presents a great challenge because it is widely used in expansion planning of urban areas, these densities can be related to the type of expansion planning that is desired, electric, population, water, or any other. It is presented a methodology for forecasting urban areas growth in high resolution, which improves the visualization, analysis and inference of density information for planning in the near future. Such methodology converts the input data into high-resolution grid maps, which in turn are divided into larger regions, where each of these regions will have its expected growth defined by convolution matrices and weighting factors, which search for historical characteristics of each region. Obtaining the best growth characteristics of the regions comes through the processing of the imperialist competitive algorithm (ICA) that searches for the best set of parameters, where these parameters are composed by the values of the convolution matrices and the weighting factors and will be used to forecast the growth of the region. The nature of the data used in the tests directs the forecasting for electric load density, so it is possible to obtain a spatial forecast of growth in high resolution and greater precision that, in this case, can be used as an important factor in the planning of load expansion, and can even be used in the planning and application of the concepts of smartgrids in the defined urban area.

 \end{otherlanguage*}
\end{resumo}