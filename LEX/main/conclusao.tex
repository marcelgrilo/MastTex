% ----------------------------------------------------------------
% Das conclusões tiradas
% ----------------------------------------------------------------
\chapter{Conclusão}
\label{conclusao}

%- explicar as CONTRIBUIÇÕES da proposta
%  - levantamento das qualidades dos resultados em relação a metodologia original
%  - mostrar que a evolução de lehman sobre a seleção dos nichos, aqui proposta foi recompensada com resultados superiores a técnica original.
%  - a remoção de nichos de baixa capacidade da busca apresentou nas análises uma superior qualidade dos indivíduos encontrados mantendo a dispersão da técnica.
%   - a proposta de seleção de nichos não resultou na perda da dispersão dos resultados, mostrando eficiente para a solução do problema inicial verificado.
%   - contribuição do pacote computacional genérico da ferramenta proposta opensource GLT (use github como referência)
%  - finalidade da proposta em relação ao problema abordado comparando com as referências bibliográficas da introdução

Este trabalho apresentou e implementou um algoritmo evolutivo capaz de obter indivíduos diversificados e com alta capacidade funcional em uma única execução. O mesmo,  com base em uma função de distância fenotípica, é capaz de manter diferentes nichos morfológicos durante a busca ao mesmo tempo que seleciona os melhores indivíduos dentro de cada um deles. Além disso, o algoritmo é eficaz em eliminar nichos com baixa capacidade funcional da busca, obtendo um conjunto final de indivíduos com melhor nota de avaliação quando comparado às técnicas nas quais se baseia.

As implementações e testes foram conduzidos dentro do contexto de geração procedural de mapas para jogos de aventura. Para tal, foram definidos o mapeamento entre genótipo e fenótipo, assim como as funções de avaliação e distância. Os testes foram realizados utilizando quatro diferentes configurações: AG, NS, Lehman e Proposta; conforme descritas no capítulo \ref{experimentos}.

Os resultados obtidos permitiram confirmar os benefícios da introdução da NS no contexto de busca por inovação, com todas as configurações baseadas na NS obtendo valores de dispersão muito superiores aos valores obtidos pela configuração baseada em AG, demonstrando que a introdução da mesma contribui significativamente para a obtenção de indivíduos diferenciados. Embora haja um grande ganho em diversificação, os testes revelam que a utilização da NS sem uma função de distância que se relacione de algum modo com a performance, não contribui com a busca por boas notas de avaliação, selecionando indivíduos somente com base na morfologia para a qual foi elaborada.

%mencionar a sessão de testes indicando os percentuais de melhora entre diversidade e avaliação

Os resultados mostram ainda, que as modificações realizadas no algoritmo apresentado em \cite{lehman2011evolving}, que foram aqui introduzidas e dão suporte a ideia de exclusão de nichos de baixa capacidade, obtiveram bons resultados em relação à qualidade dos indivíduos encontrados. Quando comparados na seção \ref{discussao_analise_e_testes}, os resultados obtidos pela configuração Lehman e Proposta, revelam um ganho de 18,80\% em relação à nota de avaliação média do conjunto $C_f$ obtido pela segunda, ao mesmo tempo que mantém o valor de dispersão com uma variação de apenas -3,09\%. Estes valores comprovam a eficiência da remoção de nichos ruins da busca para a melhoria da qualidade funcional obtida, dando prioridade a nichos com alta capacidade durante a seleção, sem que a diversidade introduzida pela NS seja prejudicada.

Nos testes realizados, a configuração Proposta foi capaz de obter, de forma fidedigna, uma quantidade significativa de mapas como solução em todas as execuções. Além disso, se comparados com os mapas obtidos pela AG, a configuração Proposta demonstra ser capaz de, em uma única execução, encontrar mapas visualmente e estruturalmente bem diversificados. Outra consequência disso, conforme discutido na seção \ref{tempo_de_execucao}, é que mesmo obtendo um tempo total de execução maior que a AG, o tempo médio por solução obtida pela proposta também se torna um avanço em relação a capacidade computacional da solução do problema proposto.

Com isso, as técnicas e algoritmos aqui propostos, evidenciam ser viáveis para o contexto de geração procedural de conteúdo onde se deseja evitar a baixa qualidade dos itens gerado, assim como repetição ou inconsistência nos padrões dos mesmos. De forma análoga, as técnicas aqui apresentadas também podem ser de boa utilidade para geração de opções diferentes em um sistema de auxílio à decisão.

Este trabalho gerou como produto uma ferramenta funcional para geração de mapas para jogos de aventura, que pode ser utilizado tanto para criação de conteúdo dinâmico quanto para auxílio no processo de \emph{Level Design}.

Fazem parte da contribuição gerada por este trabalho: o pacote computacional genérico da ferramenta proposta; assim como as extensões para utilização da NS, MOEAs e NSGA-II para a \emph{framework} AForge.NET \cite{kirillov2013aforge}, com  código fonte disponível online em \url{https://github.com/lexmelotti/nichesearch} sob licença GNU LGPLv3.

\section{Trabalhos Futuros}
\label{trabalhos_futuros}

%- continuidade da pesquisa durante alguns anos
%  - aprimoramento de nichos através de um sub algorítimo de otimização com barreiras, evitando a convergência global
%     - barreiras criadas através de k-means ou regiões limitadas pelas mediana das centroides
     
%  a utilização dinâmica de operadores genéticos e dispersão pela análise dos resultados ao longo das gerações 
  
%  operadores genéticos = cruzamento, mutação, clonagem, movimento direcionado
%  operadores de dispersão: distancia euclidiana, manhantam, ...

Como sugestão de estudos de continuidade, propõe-se uma extensão do algoritmo apresentado que aprimore a busca dentro dos nichos obtidos, efetuando uma pesquisa por  indivíduos com altas notas de avaliação limitada por barreiras, evitando assim a convergência global da busca e mantendo a diversidade obtida pelo algoritmo aqui apresentado. Para tal, sugere-se a utilização de um algoritmo genético com barreiras. Utilizando os indivíduos encontrados em $C_f$ como base para o particionamento, e técnicas de agrupamento como \emph{k-means} e \emph{k-medoids}, a busca pode ser delimitada por regiões definidas pelas medianas dos centróides e medóides obtidos. Outra opção possível, uma vez que o algoritmo aqui apresentado já realiza uma segregação por nichos morfológicos, é limitar a busca ao redor dos indivíduos de $C_f$ com o cálculos de medianas entre os mesmos.

Outra sugestão de continuidade é a utilização de operadores dinâmicos genéticos e de dispersão, com alteração baseada na análise dos resultados obtidos ao longo das gerações.

Sugere-se ainda a utilização da estrutura de dados \emph{QuadTrees} em trabalhos futuros para otimização do cálculo da busca por vizinhos mais próximos, assim como a possibilidade da utilização de outras métricas para o cálculo de distância entre mapas, podendo considerar a utilização de técnicas de similaridade de imagens, comparação entre os caminhos gerados, e outros cálculos utilizando o genótipo para simplificação e otimização.

%Sugere-se ainda a aplicação do algoritmo, aqui definido, em outros contextos onde o auxílio de tomada de decisões pode ser aplicado, como, por exemplo, o mapeamento e geração de redes de distribuição elétrica.