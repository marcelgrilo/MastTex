\begin{resumo}

Com a crescente demanda de ambientes virtuais diferenciados surge a necessidade da geração dinâmica de conteúdo, agilizando o desenvolvimento, reduzindo custos de produção e ampliando a vida útil do produto. Em termos gerais, mundos procedurais geram uma gama maior de detalhamento no ambiente e personalizam a experiência, sem prejudicar a performance do sistema, mas isso só é garantido quando a qualidade do conteúdo obtido satisfaz o usuário em qualidade e diversidade. Desta forma, é necessária a utilização de algoritmos que sejam capazes de garantir a qualidade da experiência gerada sem faltar com a diversidade. Este trabalho apresenta uma técnica que pode ser utilizada no desenvolvimento de ambientes virtuais tanto em sua produção, oferecendo ao desenvolvedor possibilidades distintas de design para serem elaboradas e refinadas, como também durante a execução, construindo simulações únicas enquanto sendo utilizada pelo usuário. Assim, a técnica desenvolvida utiliza nichos morfológicos para garantir diversidade de soluções, produzindo resultados que apresentam as melhores soluções dentro de cada grupo, superando em qualidade as técnicas existentes como o busca inovativa e competição local.

\end{resumo}

\begin{abstract}

With the increasing demand for differentiated virtual environments, the need for dynamic content generation arises, speeding up development, reducing productions costs and increase the product shelf life. In general, procedurally generated worlds, offer a greater range of details in the environment and experience customization without sacrificing performance, but this is only guaranteed when the quality of the content obtained satisfies the user in quality and diversity. Thus, the use of algorithms that are able to guarantee the quality of the generated experience without lacking diversity are required. This paper presents a technique that can be used to develop virtual environments both in its production, providing the developer with different design possibilities to be developed and refined, as well as execution, building unique simulations while interfacing with the user. Therefore, the technique developed makes use of morphological niches to ensure diversity of solutions, producing results that provide the best solutions within each group, surpassing in quality the existing techniques such as novelty search and local competition.

\end{abstract}